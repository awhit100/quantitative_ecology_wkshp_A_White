\documentclass[]{article}
\usepackage{lmodern}
\usepackage{amssymb,amsmath}
\usepackage{ifxetex,ifluatex}
\usepackage{fixltx2e} % provides \textsubscript
\ifnum 0\ifxetex 1\fi\ifluatex 1\fi=0 % if pdftex
  \usepackage[T1]{fontenc}
  \usepackage[utf8]{inputenc}
\else % if luatex or xelatex
  \ifxetex
    \usepackage{mathspec}
  \else
    \usepackage{fontspec}
  \fi
  \defaultfontfeatures{Ligatures=TeX,Scale=MatchLowercase}
\fi
% use upquote if available, for straight quotes in verbatim environments
\IfFileExists{upquote.sty}{\usepackage{upquote}}{}
% use microtype if available
\IfFileExists{microtype.sty}{%
\usepackage[]{microtype}
\UseMicrotypeSet[protrusion]{basicmath} % disable protrusion for tt fonts
}{}
\PassOptionsToPackage{hyphens}{url} % url is loaded by hyperref
\usepackage[unicode=true]{hyperref}
\hypersetup{
            pdftitle={Ecology Workshop: Project Proposal},
            pdfauthor={Allison White},
            pdfborder={0 0 0},
            breaklinks=true}
\urlstyle{same}  % don't use monospace font for urls
\usepackage[margin=1in]{geometry}
\usepackage{graphicx,grffile}
\makeatletter
\def\maxwidth{\ifdim\Gin@nat@width>\linewidth\linewidth\else\Gin@nat@width\fi}
\def\maxheight{\ifdim\Gin@nat@height>\textheight\textheight\else\Gin@nat@height\fi}
\makeatother
% Scale images if necessary, so that they will not overflow the page
% margins by default, and it is still possible to overwrite the defaults
% using explicit options in \includegraphics[width, height, ...]{}
\setkeys{Gin}{width=\maxwidth,height=\maxheight,keepaspectratio}
\IfFileExists{parskip.sty}{%
\usepackage{parskip}
}{% else
\setlength{\parindent}{0pt}
\setlength{\parskip}{6pt plus 2pt minus 1pt}
}
\setlength{\emergencystretch}{3em}  % prevent overfull lines
\providecommand{\tightlist}{%
  \setlength{\itemsep}{0pt}\setlength{\parskip}{0pt}}
\setcounter{secnumdepth}{0}
% Redefines (sub)paragraphs to behave more like sections
\ifx\paragraph\undefined\else
\let\oldparagraph\paragraph
\renewcommand{\paragraph}[1]{\oldparagraph{#1}\mbox{}}
\fi
\ifx\subparagraph\undefined\else
\let\oldsubparagraph\subparagraph
\renewcommand{\subparagraph}[1]{\oldsubparagraph{#1}\mbox{}}
\fi

% set default figure placement to htbp
\makeatletter
\def\fps@figure{htbp}
\makeatother


\title{Ecology Workshop: Project Proposal}
\author{Allison White}
\date{January 10, 2020}

\begin{document}
\maketitle

\subsection{Research Statement}\label{research-statement}

I am a PhD student in the Biology Department at Florida International
University. My research is focused on the application of active
acoustics to study the population ecology of reef-associated fishes. My
long term goal is to establish a long term fishery-independent survey of
reef fishes in the southeast United States using active acoustics.
Towards this end, I am currently investigating the design- and
model-based implications of acoustic surveys over isolated fish
aggregations for common population estimates such as fish density.
``Star'' survey designs have become an increasingly popular alternative
to parallel line designs in acoustic sampling of areas with isolated
fish aggregations such as artificial reefs and spawning aggregation
sites. While traditional parallel line surveys offer better coverage of
the area surrounding a fish aggregation and less spatial autocorrelation
between transect nodes, they often require a greater number of transects
and present several practical difficulties in maneuvering tight turns.
Star surveys involve fewer transects which are arranged in alternating
directions and which all cross at the center of the aggregation site.
Star designs may be easier to maneuver and provide a higher sampling of
the targeted aggregation, but they have an inherently large spatial
autocorrelation between transect nodes which can result in biased
estimates of fish density.

\subsection{Objectives and Hypothesis}\label{objectives-and-hypothesis}

The objective of my project is to compare acoustically-derived fish
density estimates across both star and parallel line survey designs
using common model-based approaches to incorporate the spatial
autocorrelation inherent in each. Neither of these survey designs
provide the truly random samples assumed when estimating a mean fish
density of isolated fish aggregations. Parallel line surveys provide
systematic-random samples with reduced variance estimates and star
surveys produce clustered estimates with even further-reduced estimates
of variance. I hypothesize that each of these survey designs will
produce significantly biased estimates of fish density both with and
without modelling approaches to incorporate their spatial
autocorrelation. When run as a simulation, I expect the mean fish
density estimated from the star survey design to be significantly
further from the simulated true mean than the mean estimate from the
parallel line survey.

\subsection{Datasets and Statistical
Analysis}\label{datasets-and-statistical-analysis}

We sampled goliath grouper (Epinephalus itajara) spawning aggregations
at three artificial reefs off of Jupiter, Florida using both star and
parallel hydroacoustic survey designs. I have already processed the raw
data to derive estimates of grouper density from acoustic response at
each sampling event. Grouper density estimates derived from
geostatistical and general additive models (GAM) will be computed for
both survey designs, as well as cluster model-based estimates for
parallel line surveys and a concentric interval approach for star
surveys. Conditional simulations will be performed to further
investigate the influence of each design and model approach on mean
estimates of fish density.

\end{document}
